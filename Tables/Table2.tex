\documentclass[12pt,a4paper]{standalone}

\usepackage{threeparttable}
\usepackage{booktabs}
\usepackage{caption}

%Tabulator 
\newcommand\tab[1][1cm]{\hspace*{#1}}
\newcommand\tabs[1][0.8cm]{\hspace*{#1}}
\newcommand\tabu[1][0.4cm]{\hspace*{#1}}


\begin{document}

\begin{minipage}[c][24cm]{1.3 \textwidth}

\thispagestyle{empty}

\begin{table}
\centering
\caption*{\textbf{Table 2:} 2SLS with two types of clustered standard errors}

\begin{threeparttable}
\begin{tabular}{lllll}

    \toprule
    & \multicolumn{2}{c}{Clustered by municipality} & 			\multicolumn{2}{c}{Clustered by county}\\
    \cmidrule(lr){2-3}
    \cmidrule(lr){4-5}
    & First Stage & Second Stage & 	First Stage & Second 		Stage \\ 
    \midrule
 
    Refugee inflow                 & 0.4973***   &             & 0.4973***   &              \\ 
                               & (0.0616)    &             & (0.0568)    &              \\ 
Change in share of immigrants  &             & -0.3472**   &             & -0.3472**    \\ 
                               &             & (0.1560)    &             & (0.1490)     \\ 
Vacant housing rate            & -0.0005     & 0.0097      & -0.0005     & 0.0097       \\ 
                               & (0.0076)    & (0.0144)    & (0.0074)    & (0.0122)     \\ 
Unemployment rate              & -0.0292     & -0.0482     & -0.0292     & -0.0482      \\ 
                               & (0.0229)    & (0.0345)    & (0.0253)    & (0.0324)     \\ 
Change in unemployment rate    & 0.0102      & 0.0320      & 0.0102      & 0.0320       \\ 
                               & (0.0309)    & (0.0419)    & (0.0337)    & (0.0459)     \\ 
Change in tax base             & -0.0006     & -0.0018*    & -0.0006     & -0.0018**    \\ 
                               & (0.0008)    & (0.0010)    & (0.0008)    & (0.0007)     \\ 
Change in population size      & -0.0175**   & -0.0092     & -0.0175     & -0.0092      \\ 
                               & (0.0076)    & (0.0085)    & (0.0103)    & (0.0065)     \\ 
Change in welfare spending     & 0.0091      & -0.0077     & 0.0091      & -0.0077      \\ 
                               & (0.0105)    & (0.0148)    & (0.0101)    & (0.0102)     \\ 
Small-sized population         & -0.0737*    & -0.0494     & -0.0737     & -0.0494      \\ 
                               & (0.0437)    & (0.0637)    & (0.0475)    & (0.0591)     \\ 
Large-sized population         & 0.4141***   & 0.2217      & 0.4141***   & 0.2217       \\ 
                               & (0.0963)    & (0.1443)    & (0.0892)    & (0.1610)     \\ 
Socialist majority             & 0.0392      & 0.0952      & 0.0392      & 0.0952*      \\ 
                               & (0.0441)    & (0.0718)    & (0.0389)    & (0.0470)     \\ 
Green Party                    & 0.0097      & 0.0910      & 0.0097      & 0.0910       \\ 
                               & (0.0387)    & (0.0827)    & (0.0424)    & (0.0837)     \\ 
New Democrats                  & 0.0574      & 0.0630      & 0.0574      & 0.0630       \\ 
                               & (0.0564)    & (0.0774)    & (0.0461)    & (0.0641)     \\ 
Panel 1988/91                  & 0.1807**    & -0.3422**   & 0.1807      & -0.3422**    \\ 
                               & (0.0848)    & (0.1324)    & (0.1490)    & (0.1330)     \\ 
Panel 1991/94                  & -0.2177     & 0.0393      & -0.2177     & 0.0393       \\ 
                               & (0.2074)    & (0.3028)    & (0.2054)    & (0.2560)     \\ 
Constant                       & 0.2579**    & 0.3031      & 0.2579      & 0.3031*      \\ 
                               & (0.1267)    & (0.1982)    & (0.1759)    & (0.1653)     \\ 
  % The table fragment

    \bottomrule
\end{tabular}

    \begin{tablenotes}
    \item NOTES:
    Sample size is 1917.  % The note on sample size. 
	\par
	\begingroup
	\leftskip=0.3cm % Parameter anpassen
	\noindent 
	The dependent variable in the first stage is 				'Change in share of immigrants', the dependent 				variable in the second stage is 'Changes in 				Preferences for Redistribution'. \\
    Clustered standard errors are shown in parentheses. \\
    $ \ast $ \tab significant at the 10 percent level \\
    $ \ast \ast $ \tabs significant at the 5 percent level 	\\
    $ \ast \ast \ast $ \tabu significant at the 1 percent 		level.
	\par
	\endgroup




    \end{tablenotes}
\end{threeparttable}
\end{table}

\end{minipage}
\end{document}